\section{WYSIWYG}

A maioria dos softwares responsáveis por processamento de palavras utilizam uma interface que permite que o usuário veja algo muito similar ao resultado final enquanto o documento está sendo criado. Essa interface pode ser chamada de WYSIWYG (``What You See Is What You Get'', ou em uma tradução literal, ``O que você vê é o que você obtém''). Um exemplo de programa que se encaixa nesta classificação é o \textbf{MS Word} ou \textbf{LibreOffice}.

Em editores WYSIWYG, pedaços de códigos são inseridos no documento para indicar onde a fonte deve mudar de tamanho, onde usar itálico, ou negrito, etc. Esses pedaços de código não são vistos pelo usuário em momento algum, portanto só é possível editar esse arquivo utilizando o próprio software que o criou. Com o \LaTeX, o documento consiste de apenas um arquivo de texto, que pode ser modificado com qualquer editor. É possível dizer, então, que WYSIWYW (``What You See Is What You Want'', ou ``O que você vê é o que você quer'').

\subsection{\LaTeX\ vs WYSIWYG}%Por que o \LaTeX é melhor que softwares WYSIWYG?} 
Embora existam diversas situações onde a utilização de um programa é preferida à de outro por inúmeras razões, existem alguns motivos que podem servir como motivação na escolha do \LaTeX\ como sua ferramenta para criação de textos.

\begin{itemize}
\item Velocidade.

Qualquer ação que requer um mouse ou cliques para acesso de menus será mais lento do que digitar algumas teclas do teclado. Isso significa que digitar caracteres especiais, como uma letra grega, ou principalmente funções serão muito mais rápidos no \LaTeX.

Na edição de arquivos muito grandes, e com muitas figuras e equações, o MS Word, por exemplo, se torna mais lerdo, já que ações como mover a barra de rolagem de arquivos pesados requerem muito do processador e da memória. Editar um simples arquivo de texto já não causa tantos problemas para a máquina e a modificação pode ser feita rapidamente.

\item Segurança.

Os arquivos dos editores comuns são armazenados em uma forma binária. Se este arquivo for corrompido por qualquer motivo, o usuário pode perder muitas horas de trabalho. Com arquivos de texto, como no \LaTeX, o arquivo pode ser recuperado de maneira mais fácil e com chances maiores de sucesso.

\item Separação de conteúdo e formatação.

A separação em seções e subseções de texto no \LaTeX\ torna muito mais fácil para o escritor se concentrar no texto e na sua ordem de apresentação do que em detalhes superficiais como tamanhos de fonte e estilos.

\item Integração com Sistema de Controle de Versões.

  O \LaTeX\ propicia o uso de sistemas de controles de versão, já que seu formato em texto puro pode ser gerenciado de forma simples por estes sistemas. Desta forma a cooperação entre múltiplos usuários durante a escrita de um texto se torna mais fácil.

\item Controle

Alguns editores WYSIWYG possuem configurações prévias que almejam facilitar a digitação de um texto e nestes casos os editores agem sem permissão do usuário, por exemplo: auto-capitalizando as primeiras palavras de uma frase ou selecionando automaticamente sentenças inteiras. Muitas vezes essas edições são indesejadas como no caso dos sinais de subtração, hífen e travessão. Os símbolos -, -- e --- representam três coisas diferentes e quando detalhes deste tipo fazem diferença no contexto da frase, como é frequente em textos científicos, o editor de texto acaba se tornando um problema.

\item Flexibilidade

Por possuir um formato de texto simples, o \LaTeX\ permite a utilização do vasto ferramental que está disponível para a manipulação de arquivos de texto, como por exemplo, substituição usando expressões regulares. Além disso, este formato permite que um arquivo .tex seja transportado de forma flexível e compilado em qualquer ambiente que possua o \LaTeX\ instalado. Modificações no texto também são bastante simples já que o sistema faz a numeração de seções automaticamente, bem como indexação de termos e geração de sumário.

\end{itemize}

Estas são apenas algumas das vantagens deste sistema sobre os editores de texto convencionais e muitos outros detalhes existem, principalmente quanto mais técnico é o conteúdo da escrita. Porém, não seria correto afirmar que o \LaTeX\ é melhor em todos os sentidos que editores WYSIWYG.

Como exemplo, tem-se algumas ferramentas para edição colaborativa de texto. Essas podem ser mais eficientes devido a facilidade de marcação de texto e aos efeitos visuais proporcionados. Em várias situações, a simplicidade de operação dos editores convencionais faz com que eles sejam melhores em certas tarefas, como quando se deseja fazer um protótipo de design. As interações estéticas são muito mais fáceis de serem percebidas no editor WYSIWYG, porém a manutenção do conteúdo de acordo com o design se torna mais difícil. A escolha, portanto de uma ferramenta certa para a ocasião e para a aplicação que se deseja é importante.

