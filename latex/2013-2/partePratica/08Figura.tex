% mostra conceitos sobre as figuras e environment de figuras

\documentclass{article}

\usepackage[utf8]{inputenc}
\usepackage{graphicx}
\usepackage[brazil]{babel}

\begin{document}
	\section{Introdução}
		Documento exemplo para consultas futuras do \LaTeX.

	\section{Acentuação}		
		Ele será incrementado a cada etapa nesse minicurso super legal.
	
	\section{Negrito, Itálico, Ênfase}
		A codificação no \textbf{Linux} deve \huge sempre \normalsize ser utilizada em \textbf{UTF-8} para acentuação \emph{correta}. No \textit{Windows} a \emph{maioria} dos casos utiliza-se o \textit{Latin1}.
		
	\section{\textit{Environment}}
		Nesta seção mostram-se alguns ambientes disponíveis para uso no \LaTeX. Por exemplo na subseção \ref{verbatim} o ambiente \textsf{verbatim} não interpretará inserido.

		\subsection{\textit{Enumerate}}
			\begin{enumerate}
				\item	O \LaTeX provê uma forma diferente para numeração;
				\item	Como sempre focamos no conteúdo, não na formatação;
				\item[A.]	Podemos usar outra letra também.
			\end{enumerate}
	
		\subsection{\textit{Itemize}}
			Só mudamos uma palavra para transformar o texto:
			\begin{itemize}
				\item	O \LaTeX provê uma forma diferente para numeração;
				\item	Como sempre focamos no conteúdo, não na formatação;
				\item[A.]	Podemos usar outra letra também.
			\end{itemize}
	
		\subsection{\textit{Description}}
			Algumas siglas citadas na apresentação:
			\begin{description}
				\item[ACM] \textit{Association for Computing Machinery}
				\item[IEEE]	\textit{Institute of Electrical and Electronic Engineers}
				\item[SBC] Sociedade Brasileira da Computação
			\end{description}

		\subsection{\textit{Verbatim}}
			\label{verbatim}\verb|\textbf{Porquê esse texto não fica em negrito?}|
	
	\section{Figuras}
		% Falar sobre environment;
		% Tradeoff entre caption e environment
		% Pacote {babel}
		\begin{figure}[h]
			\centering
			\label{pinto}
			\includegraphics{chick.png}
			\caption{O pintinho amarelinho}
		\end{figure}
\end{document}