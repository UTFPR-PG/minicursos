\section{Conclusão e dicas finais}
Brevemente apresentou-se a parte lógica de como o \LaTeX\space funciona, qual seu foco principal, onde e porque utilizar, instalação básica e suas diferenças entre editores comuns de texto WYSIWYG. Como apontado inicialmente este material não é de nenhuma forma ostensivo e deve ser utilizado apenas de maneira introdutória.

Aprofundamentos podem ser encontrados em \url{http://tug.org/}, enquanto que maiores detalhes da produção de texto podem ser explorados em \newline\url{http://en.wikibooks.org/wiki/LaTeX}. Para dúvidas gerais, erros de compilação e questões similares o site \tex\space da \textsf{StackExchange} \url{http://tex.stackexchange.com/} possui grande base de questões.

Nós da UTFPR-PG também possuímos uma lista de email para dúvidas em gerais sobre Software Livre e essa pode ser acessada em \newline\url{https://groups.google.com/forum/#!forum/psl_pg} ou simplesmente mandando um email para \url{psl_pg@gmail.com}. Todo o material do minicurso pode ser acessado em \url{https://bitbucket.org/fabianorosas/minicursos-utfpr/overview}