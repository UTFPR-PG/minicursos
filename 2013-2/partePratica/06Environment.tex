% mostra conceitos sobre as environment disponíveis

\documentclass{article}

\usepackage[utf8]{inputenc}

\begin{document}
	\section{Introdução}
		Documento exemplo para consultas futuras do \LaTeX.

	\section{Acentuação}		
		Ele será incrementado a cada etapa nesse minicurso super legal.
	
	\section{Negrito, Itálico, Ênfase}
		A codificação no \textbf{Linux} deve \huge sempre \normalsize ser utilizada em \textbf{UTF-8} para acentuação \emph{correta}. No \textit{Windows} a \emph{maioria} dos casos utiliza-se o \textit{Latin1}.
		
	\section{\textit{Environment}}

		\subsection{\textit{Enumerate}}
			\begin{enumerate}
				\item	O \LaTeX provê uma forma diferente para numeração;
				\item	Como sempre focamos no conteúdo, não na formatação;
				\item[A.]	Podemos usar outra letra também.
			\end{enumerate}
	
		\subsection{\textit{Itemize}}
			Só mudamos uma palavra para transformar o texto:
			\begin{itemize}
				\item	O \LaTeX provê uma forma diferente para numeração;
				\item	Como sempre focamos no conteúdo, não na formatação;
				\item[A.]	Podemos usar outra letra também.
			\end{itemize}
	
		\subsection{\textit{Description}}
			Algumas siglas citadas na apresentação:
			\begin{description}
				\item[ACM] \textit{Association for Computing Machinery}
				\item[IEEE]	\textit{Institute of Electrical and Electronic Engineers}
				\item[SBC] Sociedade Brasileira da Computação
			\end{description}

		% verbatim environment
		\verb|\textbf{Porquê esse texto não fica em negrito?}|
\end{document}