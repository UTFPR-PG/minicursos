\documentclass[12pt,openright,oneside,chapter=TITLE,section=TITLE,
	%subsection=TITLE,
	%subsubsection=TITLE,
	brazil]{utfpr-pg}

%\usepackage{cmap}
%\usepackage[T1]{fontenc}
%\usepackage{lastpage}
%\usepackage{indentfirst}
%\usepackage{color}
%\usepackage{graphicx}
%\usepackage{lipsum}

\usepackage[brazilian,hyperpageref]{backref}
\usepackage[alf]{abntex2cite}

% % Configurações do pacote backref
% % Usado sem a opção hyperpageref de backref
% \renewcommand{\backrefpagesname}{Citado na(s) página(s):~}
% % Texto padrão antes do número das páginas
% \renewcommand{\backref}{}
% % Define os textos da citação
% \renewcommand*{\backrefalt}[4]{
% 	\ifcase #1 %
% 		Nenhuma citação no texto.%
% 	\or
% 		Citado na página #2.%
% 	\else
% 		Citado #1 vezes nas páginas #2.%
% 	\fi}%

%  Informações de dados para CAPA e FOLHA DE ROSTO
%\departamento{Departamento Acadêmico de Informática}
\curso{Bacharelado em Ciência da Computação}
\titulo{Pequeno exemplo de uso da classe de TCC da UTFPR}
\autor{Calouros 2013.02}
\local{Brasil}
\data{2013}
\orientador{Fabiano Rosas}
\coorientador{Gabriel B. Casella}
\tipotrabalho{Minicurso \LaTeX}
% O preambulo deve conter o tipo do trabalho, o objetivo, 
% o nome da instituição e a área de concentração 
\preambulo{Modelo canônico de trabalho monográfico acadêmico em conformidade com
as normas ABNT apresentado à comunidade de usuários \LaTeX da UTFPR.}

% % informações do PDF
% \makeatletter
% \hypersetup{
%      	%pagebackref=true,
% 		pdftitle={\@title}, 
% 		pdfauthor={\@author},
%     	pdfsubject={\imprimirpreambulo},
% 	    pdfcreator={LaTeX with abnTeX2},
% 		pdfkeywords={abnt}{latex}{abntex}{abntex2}{trabalho acadêmico}, 
% 		colorlinks=true,       		% false: boxed links; true: colored links
%     	linkcolor=blue,          	% color of internal links
%     	citecolor=blue,        		% color of links to bibliography
%     	filecolor=magenta,      		% color of file links
% 		urlcolor=blue,
% 		bookmarksdepth=4
% }
% \makeatother

% O tamanho do parágrafo é dado por:
%\setlength{\parindent}{1.3cm}

% Controle do espaçamento entre um parágrafo e outro:
%\setlength{\parskip}{0.2cm}  % tente também \onelineskip

%\makeindex

\begin{document}
% Retira espaço extra obsoleto entre as frases.
\frenchspacing 
 \pretextual
\imprimircapa
% % Folha de rosto (o * indica que haverá a ficha bibliográfica)
 \imprimirfolhaderosto*

 \begin{resumo}
  Segundo a [3.1-3.2]{NBR6028:2003}, o resumo deve ressaltar o
  objetivo, o método, os resultados e as conclusões do documento. A ordem e a extensão
  destes itens dependem do tipo de resumo (informativo ou indicativo) e do
  tratamento que cada item recebe no documento original. O resumo deve ser
  precedido da referência do documento, com exceção do resumo inserido no
  próprio documento. (\ldots) As palavras-chave devem figurar logo abaixo do
  resumo, antecedidas da expressão Palavras-chave:, separadas entre si por
  ponto e finalizadas também por ponto.

  \vspace{\onelineskip}
    
  \noindent
  \textbf{Palavras-chaves}: latex. abntex. utfpr.
 \end{resumo}

% \pdfbookmark[0]{\contentsname}{toc}
% \tableofcontents*
% \cleardoublepage

% \textual

 \chapter{Introdução}

 Este documento e seu código-fonte são exemplos de referência de uso da classe
 \textsf{abntex2} e do pacote \textsf{abntex2cite}. O documento 
 exemplifica a elaboração de trabalho acadêmico (tese, dissertação e outros do
 gênero) produzido conforme a norma da UTFPR.
 
 Para uso da classe é necessário ter o pacote \textsf{abntex2} instalado e utilizar o arquivo \textsf{utfpr-pg.cls} como classe de documento. Os parâmetros e pacotes mais importantes vão no \verb|\documentclass[]{}| e no preâmbulo. As opções mostradas já estão de acordo com as normas e são mostradas abaixo:
 \begin{verbatim}
\documentclass[12pt,openright,oneside,chapter=TITLE,section=TITLE,
	brazil]{utfpr-pg}
	\usepackage[brazilian,hyperpageref]{backref}
\usepackage[alf]{abntex2cite}
 \end{verbatim}
 \begin{description}
	\item[12pt]	Define o tamanho da fonte;
	\item[openright]	O capítulo deve começar sempre na página da direita;
	\item[oneside]	Impressão em um lado da folha;
	\item[chapter=TITLE]	Define que o capítulo será em caixa alta;
	\item[section]	Define que a seção será em caixa alta;
	\item[brazil]	Parâmetro para a classe abntex2 sobre codificação do país;
	\item[backref]	Pacote para referências com hyperlink;
	\item[abntex2cite]	Normas para apresentação das citações no formato ABNT. O parâmetro \verb|[alf]| define citações no formato alfanumérico.
 \end{description}


% % ---
% % Finaliza a parte no bookmark do PDF, para que se inicie o bookmark na raiz
% % ---
% \bookmarksetup{startatroot}% 

% \chapter{Conclusão}
% \lipsum[31-33]

% \postextual

% % Referências bibliográficas
\nocite{Knuth:1997:ACP:260999,Knuth:1997:ACP:270146,Knuth:1998:ACP:280635}
\bibliography{referencias}

% %\glossary
% \printindex

\end{document}
