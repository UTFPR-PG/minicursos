\documentclass{handout_utfpr}

\usepackage{xspace}
\usepackage{pbox}
\usepackage[brazil]{babel}
\usepackage{xcolor}
\definecolor{dark-gray}{gray}{0.20}
\definecolor{light-gray}{gray}{0.85}

\newcommand{\com}[1]{
\colorbox{light-gray}{\texttt{\pbox{\textwidth}{\$ #1}}}
}

\handoutdate{\today}

\begin{document}
\maketitle
\section{Introdução}



\section{Comandos da CLI}

Como um guia, todas as linhas de comando terão o caracter \$ no começo, apenas para indicar a linha onde será digitado o comando em questão. O caracter deve ser ignorado.\\

Para saber em qual pasta você está trabalhando na linha de comando, digite \textbf{pwd}, que significa ``print working directory''.\\

\com{pwd\\/home/usuario}\\

A mudança de diretórios, ou seja, pastas, deve ser feita através do comando \textbf{cd} (``change directory''). Há alguns atalhos que podem ser utilizados na mudança de diretórios: o \textbf{$\sim$} é um atalho para o ``/home/usuario'', um \textbf{.} significa o diretório atual e \textbf{..} é equivalente ao diretório anterior.\\

\com{cd /home/usuario/Área\textbackslash\xspace de\textbackslash\xspace Trabalho\\\$ cd ..\\\$ pwd\\/home/usuario\\\$ cd ./Área\textbackslash\xspace de\textbackslash\xspace Trabalho\\\$ pwd\\/home/usuario/Área\textbackslash\xspace de\textbackslash\xspace Trabalho}\\

Você pode criar uma pasta usando o \textbf{mkdir} (``make directory''). Lembre-se que sempre que você não souber em que pasta você está, você pode utilizar o pwd.\\

\com{cd /tmp/\\\$ mkdir NovaPasta\\\$ mkdir NovaPasta/Pasta2}\\

Se você desejar saber o conteúdo de uma pasta, basta utilizar o comando \textbf{ls} (``list'').






\end{document}
