\documentclass[12pt]{article}
\usepackage{sbc-template}
\usepackage{graphicx,url}
\usepackage[brazil]{babel}        % pacote portugues brasileiro
\usepackage[utf8]{inputenc} % pacote para acentuacao direta
\usepackage[T1]{fontenc}        % pacote para acentuacao no pdf 
\usepackage{array}
%\usepackage{listings}
\newcolumntype {C}[1]{>{\centering\let\newline\\\arraybackslash\hspace{0pt}}m{#1}}

%\sloppy

\title{Comparação básica entre as normas ABNT, IEEE e SBC para elaboração de artigos}

\author{Gabriel B. Casella, Lin Y. Han, Weverton Carvalho}


\address{Departamento de Informática -- Universidade Tecnológica Federal do Paraná - UTFPR\\
  \email{\{gbc921, nilyuhan, wevertoncarvalho34\}@gmail.com}
}
\begin{document} 

\maketitle

\begin{resumo}
Este artigo faz uma comparação entre as três diretrizes mais utilizadas para a elaboração de artigos científicos no Brasil. ABNT, IEEE e SBC. As regras para a produção de artigos científicos são brevemente apresentadas e comparadas através de uma tabela.
\end{resumo}

\section{Introdução}	
Norma pode ser definida como um documento onde se contém especificação técnica ou outros critérios precisos desenvolvidos para serem utilizados consistentemente com uma regra, diretriz ou definição. Logo ela proporciona uma maior simplicidade, aumentando a confiabilidade e a efetividade na elaboração de um artigo. As mesmas normalmente são desenvolvidas por um órgão oficial voltada para tal.

O artigo cientifico é estruturado  por partes pré-textuais, textuais e pós-textuais, sendo algumas dessas partes consideradas obrigatórias e outras opcionais. Em pré-textuais encontram se título e subtítulo se houver, nome(s) do(s) autor(es), resumo na língua do texto, palavras-chave na língua do texto. Os textuais constituem se em  introdução, o desenvolvimento e a conclusão. Elementos pós-textuais são constituídos de: título e subtítulo (se houver), resumo em língua estrangeira, palavras-chave em língua estrangeira,nota(s) explicativa(s), referências, glossário, apêndice(s), anexo(s).

Neste trabalho apresenta se as normas com suas descrições, assim como  uma comparação para ressaltar as principais características das mesmas.

\section{ABNT}
Fundada em 1948, a Associação Brasileira de Normas Técnicas (ABNT) é responsável pela normalização técnica no país, fornecendo a base necessária ao desenvolvimento tecnológico brasileiro\cite{ABNT:NBR6022}.

Trata-se de uma entidade privada, sem fins lucrativos. A ABNT é membro fundador, única e exclusiva representante no Brasil das seguintes entidades internacionais: ISO (InternationalOrganization for Standardization), da COPANT (Comissão Panamericana de Normas Técnicas) e da AMN (Associação Mercosul de Normalização)\cite{ABNT:NBR6022}.

\section{IEEE}
O \textit{Institute of Electrical and Electronics Engineers} é a maior associação profissional do mundo dedicada a inovação e excelência tecnológica em benefício da humanidade\cite{IEEEAbout}. Possui em torno de 425 mil membros em 160 países e foi fundada em 1963

As normas do IEEE visam tornar padrão as formas de publicação, criando um consenso único em que o foco está no conteúdo e não no texto. Suas políticas também são criadas com base no retorno que essa publicação terá ao autor da obra\cite{IEEECommunicationStandards}.

\section{SBC}
A SBC (Sociedade Brasileira de Computação) é a sociedade científica sem fins lucrativos, que reúne pesquisadores, professores, estudantes e profissionais da computação. Ela faz parte da SBPC (Sociedade Brasileira para o Progresso da Ciência) e da IFIP (\textit{International Federation for Information Processing}).%\cite{SBCSobre}.

A Sociedade induz o crescimento de pesquisas científicas na área de Computação, que fundamenta a formação de conhecimento e tecnologia de autenticidade brasileira. Realiza atividades de diversas natureza para a computação no Brasil, além de ter um papel importante na elaboração de políticas para o desenvolvimento de atividade de ensino, pesquisa e disseminação do conhecimento no país\cite{SBCApresentacao}.

Geralmente, nos eventos da SBC são realizadas submissões de artigos, sendo estes na formatação devida pré-estabelecida pela própria sociedade.

\bibliographystyle{sbc}
\bibliography{references.bib}
\end{document}