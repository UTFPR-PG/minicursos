\subsection{Instalação}
%\subsubsection{Versões}
De acordo com o funcionamento descrito, o \LaTeX nada mais é do que um compilador, convertendo comandos e texto no formato final de apresentação escolhido. Sendo assim, existem compiladores diferentes para cada sistema operacional e várias implementações diferentes baseadas originalmente no \LaTeX/\TeX. Para citar dois exemplos, temos Xe\TeX\cite{xetex}, que possui melhor suporte a fontes e tipografias e 
Lua\TeX que inclui a linguagem de script Lua e é baseada no \textsf{pdflatex}.

Utilizaremos basicamente nesse minicurso o \textsf{pdflatex} que caracteriza-se basicamente por ser uma extensão do \TeX. Para instalação no Windows normalmente utiliza-se a implementação MiK\TeX\cite{miktex} que pode ser baixada aqui \url{http://miktex.org/download}. Este pacote pode automaticamente baixar as os plugins (extensões do \LaTeX) que forem necessárias para funcionamento.

Para ambientes Linux a maioria das distribuições inclui um pacote a ser instalado pelo próprio gerenciador de pacotes da distro. Normalmente é utilizado o \TeX Live que agrega os compiladores, plugins, macros e fontes.

Nos sistemas Debian e derivados (Ubuntu) o \LaTeX pode ser instalado pelos pacotes \textsf{texlive}, \textsf{texlive-base} ou pelo meta-pacote \textsf{texlive-full}. Para o Ubuntu, ainda existe um \textit{script} de instalação\cite{intall-tl-ubuntu} para obter a última versão do \TeX Live (atualmente 2013). A URL de acesso com instruções de instalação pode ser acessada em \url{https://github.com/scottkosty/install-tl-ubuntu}.

Finalmente, para edição dos textos utiliza-se um editor próprio para textos \LaTeX de forma a facilitar o uso dos comandos e macros disponíveis. Dentre os vários disponíveis\cite{wiki:texEditors} utilizaremos aqui o TeXStudio\cite{texstudio} que pode ser baixado em \url{http://texstudio.sourceforge.net/}. O TeXStudio possui várias algumas funções que facilitam na edição de textos, como compilação integrada, visualizador rápido do pdf, \textit{auto-complete} de comandos, dicionário, localização de erros no \textsf{.tex}, entre outros.