\documentclass[10pt]{article}

\usepackage[brazil]{babel}   
\usepackage[utf8]{inputenc} 

\begin{document}
\section{\LaTeX\ vs WYSIWYG}

A maioria dos softwares responsáveis por processamento de palavras utilizam uma interface de usuário que permite que o usuário veja algo muito similar ao resultado final enquanto o documento está sendo criado. Essa interface pode ser chamada de WYSIWYG (``What You See Is What You Get'', ou em uma tradução literal, ``O que você vê é o que você obtém''). No caso do MS Word, pedaços de códigos são inseridos no documento para indicar onde a fonte deve mudar de tamanho, onde usar itálico, ou negrito, etc. Esses pedaços de código não são vistos pelo usuário em momento algum, portanto só é possível editar esse arquivo utilizando o próprio Microsoft Word, um software proprietário e pago. Com o \LaTeX, o documento consiste de apenas um arquivo de texto, que pode ser modificado com qualquer editor. O código inserido pelo Word, agora é inserido pela própria pessoa que digita o texto. É possível dizer, então, que com ferramentas como o \LaTeX, WYSIWYW (``What You See Is What You Want'', ou ``O que você vê é o que você quer'').

%não sei se posso começar assim, se quiser, mude.

\subsection{Por que o \LaTeX é melhor que softwares WYSIWYG?} 

\begin{itemize}
\item Velocidade.

Qualquer coisa que requer um mouse ou cliques para acesso de menus será mais lento do que digitar algumas teclas do teclado. Isso significa que digitar caracteres especiais, como uma letra grega, ou principalmente funções serão muito mais rápidos no \LaTeX.

Na edição de arquivos muito grandes, e com muitas figuras e equações, o MS Word, por exemplo, se torna muito mais lerdo, já que ações como mover a barra de rolagem de arquivos pesados requerem muito do processador e da memória. Editar um simples arquivo de texto já não causa tantos problemas para a máquina e a modificação pode ser feita rapidamente.

\item Segurança.

O MS Word armazena seus arquivos em uma forma binária. Se este arquivo for corrompido por qualquer motivo, o usuário pode perder muitas horas de trabalho e nunca mais ter acesso ao seu arquivo. Com arquivos de texto, como no \LaTeX, se um editor falhar, basta abrir o arquivo com outro.

\item Separação de contexto e formatação.

A separação em seções e subseções de texto no \LaTeX\ torna muito mais fácil para o escritor se concentrar no texto e na sua ordem de apresentação do que em detalhes superficiais como tamanhos de fonte e estilos.

\item Integração com Sistema de Controle de Versões.

Para sistemas de controles de versão, que contam com uma colaboração de vários usuários que precisam integrar o seu projeto em um só, o \LaTeX\ se torna essencial simplesmente pelo seu texto simples, dessa forma também armazenando pouco espaço no servidor. A comparação de dois arquivos usando o ``diff'' também pode ser realizada muito mais facilmente do que no Word, por exemplo.

\item Controle

Muitas vezes, um editor WYSIWYG tentará ser mais esperto que você. Por exemplo: auto-capitalizando as primeiras palavras de uma frase, ou automaticamente selecionar sentenças inteiras. Muitas vezes esse tipo de editor tenta inferir o que você quer -- e acaba errando. É o próprio caso do traço. Há diferenças entre os usos de -, --,--- , e quando o editor WYSIWYG tenta acertar e erra, pode ser muito chato. O \LaTeX\ garante que o que o usuário deseja, é o que ele terá.

\item Flexibilidade

Com o \LaTeX\ é possível utilizar outras ``ferramentas'' na busca, como expressões regulares. Há também a possibilidade de se definir ``macros'' -- representações mais semânticas de um comando. 

O uso de scripts -- pequenos programas que realizam funções e não necessitam de interação do usuário -- é muito mais simples, como por exemplo exportar uma matriz para uma tabela no \LaTeX.


\end{itemize}

Certamente não é em todo caso que o \LaTeX\ é melhor que editores WYSIWYG, como quando o usuário deseja fazer o protótipo do design. As interações estéticas são muito mais fáceis de serem vistas no editor WYSIWYG, porém a manutenção do conteúdo de acordo com o design se torna muito mais difícil. 

\end{document}
